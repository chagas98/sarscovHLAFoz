\pretextualchapter{Resumen}
\reference % linha em branco depois

% O resumo em espanhol deve ser organizado em apenas um parágrafo mesmo.


La vigilancia genómica de SARS-CoV-2 ha sido fundamental para monitorear la aparición de mutaciones y entender sus implicaciones en la gravedad y propagación del virus, ayudando a prever los impactos en la eficacia de la vacunación y en la respuesta inmune de la población. En este estudio, mapeamos las variantes de SARS-CoV-2 circulantes en Foz do Iguaçu, Paraná, Brasil, entre 2020 y 2022. Nos centramos en identificar mutaciones missense, que alteran la secuencia de aminoácidos de las proteínas virales, y evaluamos su efecto en la presentación de epítopos, especialmente para los alelos HLA-B en la población local. Los resultados indicaron que los epítopos asociados con los alelos B07:02 y B27:05 presentaron una mayor pérdida de unión debido a mutaciones, y que la aparición de ciertas variantes puede estar relacionada con la proximidad de la ciudad a Paraguay. Aunque no se encontró una correlación directa entre el perfil alélico de pacientes ingresados en la UCI y la circulación de mutaciones específicas, el estudio reveló la dinámica mutacional del SARS-CoV-2 a lo largo del tiempo en Foz do Iguaçu durante la pandemia. Así, esta investigación identificó mutaciones que potencialmente afectaron la propagación del virus y proporcionó una visión general de las variantes circulantes, contribuyendo a futuras estrategias de vigilancia epidemiológica.

\noindent Palavras-clave: SARS-CoV-2; genómica; variantes; Foz do Iguaçu

